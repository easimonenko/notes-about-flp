% overview-of-classifications.tex
% Title: Обзор классификаций языков программирования
% Author: Evgeny Simonenko <easimonenko@mail.ru>
% License: CC BY-ND 4.0

\documentclass[9pt,pdf]{beamer}

\usepackage{polyglossia}
\setdefaultlanguage{russian}
\setotherlanguage{english}
\defaultfontfeatures{Ligatures={TeX},Renderer=Basic}
\setmainfont[Ligatures={TeX,Historic}]{Linux Libertine O}
\setsansfont{DejaVu Sans}
\setmonofont{Linux Libertine Mono O}

%\usetheme{Boadilla}
\usetheme{Verona} % Минималистичная.
%\usetheme{Copenhagen}
%\usetheme{Marburg} % Для широкого экрана.
%\usetheme{JLTree} % Интересный вариант.
%\usetheme{Rochester} % Интересный вариант.
%\usetheme{SaintPetersburg} % :) Прикольная, но слишком минималистичная.
\usecolortheme{dove}
%\usecolortheme{seagull}

%\usepackage{biblatex}
%\addbibresource{bibliography.bib}

\setcounter{tocdepth}{1}

\begin{document}
	
\author{Симоненко Е.А.}
\title{Языки и системы программирования}
\subtitle{Обзор классификаций языков программирования}
\institute[КубГТУ]{Кубанский государственный технологический университет}
\date{2019}
	
\begin{frame}
	\titlepage
\end{frame}
	
\begin{frame}
	\frametitle{Содержание}
	\tableofcontents
\end{frame}

\section{По областям применения}

\begin{frame}
	\frametitle{По областям применения}
	\begin{itemize}
		\item Языки для научных приложений
		\item Языки для коммерческих приложений
		\item Языки для искусственного интеллекта
		\item Языки системного программирования
		\item Языки сценариев (скриптов)
		\item Специализированные языки
	\end{itemize}
\end{frame}

\subsection{Языки для научных приложений}

\begin{frame}
	\frametitle{Языки для научных приложений}
	\begin{itemize}
		\item Fortran
		\item C, C++
		\item R
		\item Python, Julia
		\item Scilab, Maxima, GNU Octave, Axiom, Reduce
		\item Haskell, Agda
	\end{itemize}
\end{frame}

\subsection{Языки для коммерческих приложений}

\begin{frame}
	\frametitle{Языки для коммерческих приложений}
	\begin{itemize}
		\item Java, C\#
		\item C, C++
		\item Perl, Python, PHP
		\item Haskell
		\item JavaScript, TypeScript
		\item Lisp, Clojure, ClojureScript
	\end{itemize}
\end{frame}

\subsection{Языки для искусственного интеллекта}

\begin{frame}
	\frametitle{Языки для искусственного интеллекта}
	\begin{itemize}
		\item Prolog
		\item Python
		\item Lisp
		\item CLIPS
	\end{itemize}
\end{frame}

\subsection{Языки системного программирования}

\begin{frame}
	\frametitle{Языки системного программирования}
	\begin{itemize}
		\item C, C++
		\item Rust
		\item Java
	\end{itemize}
\end{frame}

\subsection{Языки сценариев (скриптов)}

\begin{frame}
	\frametitle{Языки сценариев (скриптов)}
	\begin{itemize}
		\item Shell
		\item Perl
		\item Python
		\item JavaScript
		\item Emacs Lisp
		\item Scheme
		\item R
	\end{itemize}
\end{frame}

\subsection{Специализированные языки}

\begin{frame}
	\frametitle{Специализированные языки}
	\begin{itemize}
		\item Scratch
		\item Siemens Step 7
		\item Coq, Isabelle
		\item Ant
		\item OpenCL, CUDA: C-подобный язык для графических карт.
	\end{itemize}
\end{frame}

\section{По способу написания программ}

\begin{frame}
	\frametitle{По способу написания программ}
	\begin{itemize}
		\item текстовые
		\item графические
	\end{itemize}
\end{frame}

\begin{frame}
	\frametitle{Текстовые языки}
	Подавляющее большинство языков программирования.
	
	C++, Java, Python, C\#, R ...
\end{frame}

\begin{frame}
	\frametitle{Графические языки}
	Редкий случай.
	\begin{itemize}
		\item Scratch: для обучения детей программированию.
		\item Siemens Step 7: для программирования PLC.
		\item UML: транслируется в <<обычный>> язык программирования.
	\end{itemize}
\end{frame}

\section{Императивный и декларативный подходы}

\begin{frame}
	\frametitle{Императивные языки}
	\begin{itemize}
		\item C, C++
		\item Fortran, Pascal
		\item Java, C\#
		\item Python, Perl, R
	\end{itemize}
\end{frame}

\begin{frame}
	\frametitle{Декларативные языки}
	\begin{itemize}
		\item Prolog
		\item Haskell, Agda, Idris
		\item SQL
	\end{itemize}
\end{frame}

\section{Компилируемые и интерпретируемые языки}

\begin{frame}
	\frametitle{Компилируемые языки}
	Система программирования содержит компилятор. Этап компиляции присутствует в явном виде: чтобы запустить программу, её нужно сначала скомпилировать.
	\begin{itemize}
		\item C, C++, Rust
		\item Fortran, Pascal
		\item Java, C\#
		\item Haskell, Agda, Idris
		\item TypeScript
		\item Visual Prolog
	\end{itemize}
\end{frame}

\begin{frame}
	\frametitle{Интерпретируемые языки}
	Система программирования может содержать компилятор, но для запуска программы не требуется вызывать его в явном виде.
	\begin{itemize}
		\item Shell
		\item Perl, Python, R, Ruby
		\item JavaScript
		\item Scheme, ClojureScript, Emacs Lisp
		\item SWI-Prolog, GNU Prolog
	\end{itemize}
\end{frame}

\begin{frame}
	\frametitle{Языки с компиляцией в машинный код}
	\begin{itemize}
		\item C, C++, Rust
		\item Fortran, Pascal
	\end{itemize}
\end{frame}

\begin{frame}
	\frametitle{Языки с компиляцией в байт-код}
	\begin{itemize}
		\item Java, C\#
		\item Haskell
	\end{itemize}
\end{frame}

\begin{frame}
	\frametitle{Языки с компиляцией в другой язык}
	\begin{itemize}
		\item TypeScript -> JavaScript
		\item Idris -> JavaScript
	\end{itemize}
\end{frame}

\begin{frame}
	\frametitle{Интерпретируемые языки со <<скрытой>> компиляцией в байт-код}
	\begin{itemize}
		\item Python, Ruby
		\item Perl
		\item R
		\item JavaScript
		\item Scheme, Emacs Lisp
	\end{itemize}
\end{frame}

\begin{frame}
	\frametitle{<<Подлинно>> интерпретируемые языки}
	\begin{itemize}
		\item Shell
		\item Scilab, Maxima, GNU Octave?
		\item SWI-Prolog, GNU Prolog?
	\end{itemize}
\end{frame}

\section{Низкоуровневые и высокоуровневые языки}

\begin{frame}
	\frametitle{Низкоуровневые и высокоуровневые языки}
	Когда-то языки более высокого уровня чем ассемблер считались высокоуровневыми. При таком подходе практически все распространённые языки следует считать высокоуровневыми. Более современный подход таков:
	\begin{itemize}
		\item \textit{низкоуровневые языки} оперируют объектами непосредственно связанными с архитектурой компьютера и его операционной системой, как-то: байты, биты, адреса, прерывания, системные вызовы, потоки и процессы. Программы на таких языках как-правило компилируются в непосредственно исполняемый машинный код.
		\item \textit{высокоуровневые языки} работают с абстракциями, представляющими математические или материальные объекты, например, с текстами, графами, виджетами, таблицами и записями баз данных. Такие языки обычно не имеют прямого доступа к процессору и памяти. Программы на них как-правило компилируются в байт-код или интерпретируются.
	\end{itemize}
\end{frame}

\begin{frame}
	\frametitle{Низкоуровневые языки}
	\begin{itemize}
		\item C
		\item C++
		\item Rust
	\end{itemize}
\end{frame}

\begin{frame}
	\frametitle{Высокоуровневые языки}
	\begin{itemize}
		\item Prolog
		\item Haskell, Agda, Idris
		\item Python, R, Perl, Ruby
		\item Emacs Lisp, Scheme, ClojureScript
		\item Java, C\#, Clojure
	\end{itemize}
\end{frame}

%\begin{thebibliography}{10}
%\printbibliography
%\end{thebibliography}

\section*{Благодарность}

\begin{frame}
\center

\textit{Благодарю за внимание!}

\textbf{\textsl{\inserttitle}}

\textsl{\insertsubtitle}

\insertauthor

\url{easimonenko@mail.ru}
\end{frame}

\end{document}